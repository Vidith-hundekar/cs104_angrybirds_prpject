\documentclass{article}
\usepackage{graphicx}
\usepackage{hyperref}
\usepackage{listings}
\usepackage{xcolor}

\definecolor{codegreen}{rgb}{0,0.6,0}
\definecolor{codegray}{rgb}{0.5,0.5,0.5}
\definecolor{codepurple}{rgb}{0.58,0,0.82}
\definecolor{backcolour}{rgb}{0.95,0.95,0.92}

\lstdefinestyle{mystyle}{
    backgroundcolor=\color{backcolour},   
    commentstyle=\color{codegreen},
    keywordstyle=\color{magenta},
    numberstyle=\tiny\color{codegray},
    stringstyle=\color{codepurple},
    basicstyle=\ttfamily\footnotesize,
    breakatwhitespace=false,         
    breaklines=true,                 
    captionpos=b,                    
    keepspaces=true,                 
    numbers=left,                    
    numbersep=5pt,                  
    showspaces=false,                
    showstringspaces=false,
    showtabs=false,                  
    tabsize=2
}

\lstset{style=mystyle}

\title{Angrybirds(pygame-ce)}
\author{CS-104 Project - Spring 2024-25 \\ Vidith Hundekar}
\date{}

\begin{document}

\maketitle

\section*{Outline}
This report is about the game "Angry Birds" a 2Player game. It covers the game's instructions, implementation, game aspects, gameplay description, ideas, references and the challenges encountered during development.

\tableofcontents

\section{Introduction to my Game}
This is a 2Player game, each player tries to destroy opponent's fortress with their angry birds, first to do that with better accuracy WINS.

\section{Modules}
The external modules used are:

\begin{itemize}
    \item \texttt{pygame-ce} - Version of pygame, Python module designed for writing code for games.
    \item \texttt{Random} - A module that generates random number for various distributions.
    \item \texttt{Time} - A module that provides various time-related functions.
    \item \texttt{Math} - A module that provides many mathematical functions
\end{itemize}

\section{Repo tree}
The project directory is as follows:

\begin{lstlisting}
.
├── Modules
│   ├── MainMenu.py
│   ├── PlayGame.py
│   ├── Preferences.py
│   └── Scores.py
├── media
│   ├── fonts
│   ├── images
│   ├── sounds
│   └── videos
├── data
│   ├── path.txt
│   └── LeastTimes.txt
├── game.py
└── settings.py
\end{lstlisting}

\begin{itemize}
    \item \textbf{\texttt{main.py}} - The main game loop.
    \item \textbf{\texttt{game.py}} - Contains all the global variables and modules necessary for the game to function smoothly.
    \item \textbf{\texttt{Modules}} - The programs that manage various game parts.
    \item \textbf{\texttt{media}} - Contains all the images, sounds, and fonts used in the game.
    \item \textbf{\texttt{data}} - Contains the path of the maze and the High ScoreCard (Least Time Taken).
\end{itemize}

\section{Instructions to run}
\subsection{Game Navigation}
Note: To ensure easy navigation between various screens, a back button is introduced which smoothly takes you to the previous screen.

\subsubsection{Intro Screen}
The game starts with an Intro Screen:

\begin{figure}[h]
    \centering
    \includegraphics[width=0.5\textwidth]{report/Intro.png}
    \caption{Intro Screen}
\end{figure}

\subsubsection{Main Menu}
After loading, you will be greeted with a Main Menu, from which you can choose to:

\begin{itemize}
    \item Play
    \item See the Fastest Solves in each Level
    \item Customize the Game: Mute or Unmute
    \item Quit
\end{itemize}

You can select any of these by pressing on these buttons:

\begin{figure}[h]
    \centering
    \includegraphics[width=0.5\textwidth]{report/MainMenu.png}
    \caption{Main Menu}
\end{figure}

\subsubsection{Game Level Selection}
We have three levels of mazes from which you can choose:

\begin{figure}[h]
    \centering
    \includegraphics[width=0.5\textwidth]{report/GameLevel.png}
    \caption{Game Level Selection}
\end{figure}

\subsection{Gameplay description}
The game starts, waiting for you to navigate using the arrow keys or [W A S D]. The goal is to reach the door in the opposite corner. The \textit{Score} is measured in terms of the time taken to go to the opposite end: the Lower, the Better!

Various themes can be changed using the \textit{Change Theme} button. The music can be turned off by pressing the Music button.

Some examples of the game screens:

\begin{figure}[h]
    \centering
    \includegraphics[width=0.5\textwidth]{report/MazeGame.png}
    \caption{Game Starts!}
\end{figure}

\begin{figure}[h]
    \centering
    \includegraphics[width=0.5\textwidth]{report/MazeGameThemes.png}
    \caption{Game Play}
\end{figure}

\subsubsection{Game Over}
On reaching the opposite end, the game ends, and the time taken is displayed:

\begin{figure}[h]
    \centering
    \includegraphics[width=0.5\textwidth]{report/GameOver.png}
    \caption{Game Over}
\end{figure}

\subsubsection{Fastest Solves}
On clicking the Fastest Solves button on the Main Menu, you will see the Least Time taken to solve the various levels of the maze. An example screen:

\begin{figure}[h]
    \centering
    \includegraphics[width=0.5\textwidth]{report/HighScores.png}
    \caption{Fastest Solves}
\end{figure}

\subsubsection{Preferences}
This window enables you to mute the music part of the game. You can do this by clicking on the red music button. If you want the music back on, click the button again.

\begin{figure}[h]
    \centering
    \includegraphics[width=0.5\textwidth]{report/PreferencesSoundOn.png}
    \caption{Music On}
\end{figure}

\begin{figure}[h]
    \centering
    \includegraphics[width=0.5\textwidth]{report/PreferencesSoundOff.png}
    \caption{Music Off}
\end{figure}

\subsubsection{Quit}
On clicking this button, the game ends and the program terminates.

\section{Various Implementations in the Code}
For the maze generation, I used the \textit{Recursive Backtracking} algorithm. This algorithm is a randomized version of the depth-first search algorithm. The algorithm starts at a random cell, chooses a random neighboring cell that has not been visited, creates a path between the two cells, and moves to the next cell. The algorithm continues until it has visited every cell in the grid. I have modified this algorithm slightly to make the wall size and the path size the same, which makes the maze look more appealing.

For the pathfinding, I used the A* algorithm. The A* algorithm is a pathfinding algorithm that uses a heuristic to determine the next node to visit in a graph. The algorithm uses a priority queue to determine the next node to visit based on the cost of the path to that node and the heuristic value of the node. The algorithm continues until it reaches the goal node or there are no more nodes to visit.

I have used the \texttt{heapq} module to implement the priority queue for the A* algorithm.

For the pygame functions, I referred to the official documentation of pygame and pygame-ce, mostly the latter for the updated functions and methods.

\subsection{Customizations in the Game}
A list of all the special customizations implemented in the game:

\begin{itemize}
    \item Animation is when the player moves.
    \item Dynamic Background of the Main Menu.
    \item Music and Sound.
    \item Themes for the Game.
    \item High Scores.
    \item Preferences.
    \item Back Button for easy navigation.
    \item Customized Fonts.
    \item Responsive Buttons.
\end{itemize}

\section{References}
\begin{enumerate}
    \item \href{https://www.pygame.org/docs/}{Pygame Official Documentation}
    \item \href{https://pyga.me/docs/}{Pygame CE Official Documentation}
    \item \href{https://professor-l.github.io/mazes/}{Maze Generation Algorithms by \texttt{professor-l}}
    \item \href{OtherResources/A*.md}{A* Algorithm}
\end{enumerate}

\end{document}